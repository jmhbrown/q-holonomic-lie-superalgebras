\documentclass[a4paper]{article}

\usepackage[margin=1in]{geometry}
\usepackage{amsmath,amsthm,amssymb,mathrsfs}
\usepackage{graphicx}
\usepackage[font=small]{caption} % Without options this makes it so the stray : is removed for figures with empty captions.
% \graphicspath{{./figures/}}

\usepackage[colorlinks=true,linkcolor=blue]{hyperref}
\bibliographystyle{alpha}



\usepackage{tikz}
% \usetikzlibrary{}
% \usetikzlibrary{decorations.pathmorphing} % For wiggly lines
% \usetikzlibrary{decorations.markings} % For mid-line arrows
% \usetikzlibrary{arrows.meta} % For more arrow options


% Todo stuff:
\usepackage{todonotes}
\presetkeys{todonotes}{size=\small}{}

% Theorems | Props | Defs | Etc share a counter:
\newtheorem{counter}{Counter}[section]
\newtheorem{definition}[counter]{definition}
\newtheorem{theorem}[counter]{theorem}
\newtheorem{proposition}[counter]{proposition}
\newtheorem{lemma}[counter]{lemma}
\newtheorem{remark}[counter]{remark}
\newtheorem{corollary}[counter]{corollary}

% Blackboard Bold Shortcuts
\newcommand{\C}{\mathbb{C}}
\newcommand{\Q}{\mathbb{Q}}
\newcommand{\R}{\mathbb{R}}
\newcommand{\Z}{\mathbb{Z}}


% Looks like \mathbb{1}
\newcommand{\idty}{\mathrm{1}\mkern-4mu{\mathchoice{}{}{\mskip-0.5mu}{\mskip-1mu}}\mathrm{l}}

% Operators
\DeclareMathOperator{\Tor}{Tor}
\DeclareMathOperator{\Ext}{Ext}
\DeclareMathOperator{\Hom}{Hom}
\DeclareMathOperator{\Rep}{Rep}


% My research shortcuts
% Building blocks
\newcommand{\overarrowsplus}{\raisebox{-.6ex}{\tikz[{To[sharp]}-,scale=.3]{	\draw (0,1) -- (1,0);\draw[white,-,line width= 3pt] (1,1) -- (0,0);	\draw (1,1) -- (0,0);}}}
\newcommand{\overarrowsminus}{\raisebox{-.6ex}{\tikz[{To[]}-,scale=.3]{	\draw (1,1) -- (0,0);\draw[white,-,line width= 3pt] (0,1) -- (1,0);	\draw (0,1) -- (1,0);}}}
\newcommand{\downcurvearrowleft}{\raisebox{1ex}{\scalebox{-1}{$\curvearrowright$}}}
\newcommand{\downcurvearrowright}{\raisebox{1ex}{\scalebox{-1}{$\curvearrowleft$}}}

%opening
\begin{document}
\noindent{\large\bf Project Description \hfill Count Me In 2025}
\vspace{1cm}

\section*{Overview}
This project asks what types of functions can appear in (well behaved) \emph{representations of Lie superalgebras}.
Specifically, do these functions have strong enough recursive behavior to qualify as \emph{q-holonomic functions}?

We're motivated by the fact that these representations can be used to make quantum knot invariants, whose q-holonomicity is closely linked to the role they play in physics.
Lie superalgebras are particularly interesting because they appear in recent constructions of physical theories known as Chern-Simons theories \cite{Mikhaylov_Witten_2015}.

\section*{Details}

We will focus on \emph{typical representations} of the classical lie superalgebra $\mathfrak{sl}(n|m)$, where $n \neq m$.
These are classified up to isomorphism by a tuple of complex parameters $(a_1,\ldots,a_{m+n+1})$ which must satisfy the following set of linear inequalities:
\begin{equation}
  a_{m+1} \neq \sum_{k=m+2}^j a_k - \sum_{\ell = 1}^m a_\ell -2m - 2 + i + j
\end{equation}
for $i = 1,\ldots,m+1$, $j = m+1,\ldots, m+n+1$, see \cite[Example 1, pg 620]{Kac_1978}.

The parameters $a_k$ are called the \emph{weights} of the associated representation $V(\overline{a})$, which is characterised by having a \emph{highest weight vector} $v \in V(\overline{a})$ such that
\begin{equation}
    h_k v = a_k v, \quad\text{ and }\quad E_k v = 0  \quad\text{for } k = 1,\ldots,m+n+1.
\end{equation}
\begin{itemize}
  \item We want to describe these representations in terms of explicit matrices (which will depend on the parameters $a_k$.)
  \item Then we want to prove that the coefficients of those matrices are q-holonomic functions.
  \item We'd like to also understand if the $R$-matrix has q-holonomic coefficients.
\end{itemize}



\bibliography{\jobname.bib}

\end{document}

